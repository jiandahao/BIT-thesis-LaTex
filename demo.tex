
%%==================================================
%% demo.tex for BIT Thesis
%% modified by yang yating
%% version: 1.2
%% last update: Jan. 4th, 2018
%%==================================================

% 默认单面打印 oneside 、硕士论文模板 master

\documentclass[oneside, master,normal]{BIT-thesis-grd-jdh}

% 模板选项: 硕士论文 master; 博士论文 doctor
% 正常模式:normal  自查重模式:selfSimilarCheck  盲审模式:blindCheck
% 提交学校的查重文件可以直接使用normal模式结果
% 自查重模式主要用于关闭图片、公式等内容的显示,以减少文章字符数和降低PDF转word过程中出现的乱码,节省查重费用支出。应结合\insertcontents系列命令使用。对于土豪此选项没有任何卵用。。。。。
% 盲审模式主要根据盲审文件格式要求,隐去了作者、导师、致谢等信息,更改发表论文的格式


\begin{document}

%%%%%%%%%%%%%%%%%%%%%%%%%%%%%%
%% 封面
%%%%%%%%%%%%%%%%%%%%%%%%%%%%%%

% 中文封面内容(关注内容而不是表现形式)
\classification{TQ028.1} %可参考http://www.clcindex.com/category/TN91/
\UDC{540}

\title{形状记忆聚氨酯的RapidIO合成及其在织物中的应用}
\vtitle{形状记忆聚氨酯的\makeVerticalenWords{RapidIO} 合成及其在织物中的应用}
\author{张三}
\institute{信息与电子学院}
\advisor{**教授}
\chairman{**教授}
\degree{工学硕士(博士)}
\major{电子科学与技术}
\school{北京理工大学}
\defenddate{2019年6月}
%\studentnumber{**********}


% 英文封面内容(关注内容而不是表现形式)
\englishtitle{Synthesis and Application on textile of the Shape\\Memory Polyurethane}
\englishauthor{Zhangsan}
\englishadvisor{Prof. **}
\englishchairman{Prof. **}
\englishschool{Beijing Insititute of Technology}
\englishinstitute{School of Information and Electronics}
\englishdegree{Master}
\englishmajor{Electronics Science and Technology}
\englishdate{6,2019}

% 封面绘制
\maketitle

% 中文信息
\makeChineseInfo

% 英文信息
\makeEnglishInfo

%打印竖排论文题目
\makeVerticalTitle

% 论文原创性声明和使用授权
\makeDeclareOriginal

%%%%%%%%%%%%%%%%%%%%%%%%%%%%%%
%% 前置部分
%%%%%%%%%%%%%%%%%%%%%%%%%%%%%%
\frontmatter

% 摘要
\include{chapters/abstract}
%% 符号对照表,可选,如不用可注释掉
\input{chapters/denotation}
% 加入目录
\tableofcontents


%加入图、表索引(同时取消图表索引中章之间的垂直间隔)
\let\origaddvspace\addvspace
\renewcommand{\addvspace}[1]{}
\listoffigures
\listoftables
\renewcommand{\addvspace}[1]{\origaddvspace{#1}}



%%%%%%%%%%%%%%%%%%%%%%%%%%%%%%
%% 正主体部分
%%%%%%%%%%%%%%%%%%%%%%%%%%%%%%
\mainmatter

%% 各章正文内容
%%%==================================================
%% chapter01.tex for BIT Master Thesis
%% modified by yang yating
%% version: 0.1
%% last update: Dec 25th, 2016
%%==================================================
\chapter{绪论}
\label{chap:intro}
%在下方加入各小节内容
\section{本论文研究的目的和意义}

近年来,随着人们生活水平的不断提高,人们越来越注重周围环境对身体健康的影响。作为服装是人们时时刻刻最贴近的环境,尤其是内衣,对人体健康有很大的影响。由于合时刻刻最贴近的环境,尤其是内衣,对人体健康有很大的影响。由于合成纤维的衣着舒适性、手感性,天然纤维的发展又成为人们关注的一大热点。

……\upcite{Takahashi1996Structure,Xia2002Analysis,Jiang1989,Mao2000Motion,Feng1998}
\input{chapters/chapter1/chapter1_2}


%%%%%%%%%%%%%论文正文部分%%%%%%%%%%%%%%%%%%%%%%%%%%%%%%%%%%%%%%%%
%%==================================================
%% chapter01.tex for BIT Master Thesis
%% modified by yang yating
%% version: 0.1
%% last update: Dec 25th, 2016
%%==================================================
\chapter{绪论}
\label{chap:intro}
%在下方加入各小节内容
\section{本论文研究的目的和意义}

近年来,随着人们生活水平的不断提高,人们越来越注重周围环境对身体健康的影响。作为服装是人们时时刻刻最贴近的环境,尤其是内衣,对人体健康有很大的影响。由于合时刻刻最贴近的环境,尤其是内衣,对人体健康有很大的影响。由于合成纤维的衣着舒适性、手感性,天然纤维的发展又成为人们关注的一大热点。

……\upcite{Takahashi1996Structure,Xia2002Analysis,Jiang1989,Mao2000Motion,Feng1998}
\input{chapters/chapter1/chapter1_2}


%结论
\include{chapters/conclusion}
%%%%%%%%%%%%%%%%%%%%%%%%%%%%%%%%%%%%%%%%%%%%%%%%%%%%%%%%%%%%%%%%%
%%%%%%%%%%%%以下部分仅用于举例样式%%%%%%%%%%%%%%%%%%%%%%%%%%%%%%%
接下来这段话不会被显示\fbox{\insertContents{$\tau$自查重模式下看不到的句子}}\\
\insertContents{
	If this part is shown, then the current mode is not similar self-check
	\begin{equation}
		E = mc^2
	\end{equation}
	\begin{equation}
	\begin{split}
		P_{f1} (\hat{d}) &= \sum_{n=0}^{N/4 -1}r^{*}(\hat{d}+n+N/2) r(\hat{d}+3N/4+n) \\
		&= e^{j\pi \varepsilon /2}\sum_{n=0}^{N/4 -1}s^{*}(\hat{d}+n+N/2) s(\hat{d}+3N/4+n) \\
		&= e^{j\pi \varepsilon /2}\sum_{n=0}^{N/4 -1}\left|s(\hat{d}+n+N/2)\right|^2 
	\end{split}
	\end{equation}
}
\insertEquation{
	If this part is shown, then the current mode is not similar self-check
	\begin{equation}
		E = mc^2
	\end{equation}
}
%%%%%%%%%%%%%%%%%%%%%%%%%%%%%%%%%%%%%%%%%%%%%%%%%%%%%%%%%%%%%%%%%
%% 参考文献,五号字,使用 BibTeX,包含参考文献文件.bib
%\bibliography{reference/chap1,reference/chap2} %多个章节的参考文献
\bibliography{reference/references}


%%%%%%%%%%%%%%%%%%%%%%%%%%%%%%
%% 后置部分
%%%%%%%%%%%%%%%%%%%%%%%%%%%%%%

%% 附录(章节编号重新计算,使用字母进行编号)
\appendix
\renewcommand\theequation{\Alph{chapter}--\arabic{equation}}  % 附录中编号形式是"A-1"的样子
\renewcommand\thefigure{\Alph{chapter}--\arabic{figure}}
\renewcommand\thetable{\Alph{chapter}--\arabic{table}}

\include{chapters/app1} 
\include{chapters/app2} 

%(其后部分无编号)
\backmatter

% 发表文章目录
%%==================================================
%% pub.tex for BIT Master Thesis
%% modified by yang yating
%% version: 0.1
%% last update: Dec 25th, 2016
%%==================================================

\begin{publications}{99}

%    \item\textsc{高凌}. {交联型与线形水性聚氨酯的形状记忆性能比较}[J].
%      化工进展, 2006, 532-535.(核心期刊)
	\pubitem{一}{高凌}{交联型与线形水性聚氨酯的形状记忆性能比较}{J}{化工进展, 2006, 532-535.(核心期刊)}
   %非盲审模式下显示为: 高凌. 交联型与线形水性聚氨酯的形状记忆性能比较 [J]. 化工进展, 2006, 532 -535.(核心期刊)
   %盲审模式下显示为:第一作者 + 交联型与线形水性聚氨酯的形状记忆性能比较 + 化工进展, 2006, 532- 535.(核心期刊).
\end{publications}

% 致谢
%%==================================================
%% thanks.tex for BIT Master Thesis
%% modified by yang yating
%% version: 0.1
%% last update: Dec 25th, 2016
%%==================================================

%\begin{thanks}
%这个是官方模板的做法
%本论文的工作是在导师……。
%\end{thanks}

\sayThanks{
%这是模板BIT-thesis-template-grd-jdh中新增的命令,用以实现在盲审模式下关闭这一部分的显示
%注意是使用BIT-thesis-grd-jdh.cls格式控制文件的情况下
感谢人民,感谢党。

本论文的工作是在导师……。

}

% 作者简介(博士论文需要)
\include{chapters/resume}


\end{document}
